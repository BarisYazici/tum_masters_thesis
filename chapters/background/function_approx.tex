\section{Function Approximation}
As mentioned before, it is often impossible to find either the optimal policy or the optimal value function due to computational resources. It is often enough to find approximate solutions. Various function approximators are available for us to use. The main categorization follows as; linear and nonlinear function approximators. Their working principle is similar; they both follow a specific policy and sampling process. Based on the experiences, they approximate either policy or the value function. 
% by generalizing the states it has been over the states the agent has not been so far.

The application of neural networks provided RL methods a boost.(\todo{REFERENCE}) Since the introduction of function approximators, the success of RL applications has increased significantly. Although function approximation tools bring convergency issues in some corner cases, it shows excellent success practically. 

The notation \(V(s)\) changes to \(V(s, w)\), if we define value function with weight parameterization. To find suitable weight parameters that represent the value function as general as possible, one needs to update parameters \(w\) at each step towards the smallest loss region based on a particular objective description. Firstly, we need to define an objective function and then find a method to tweak our weight parameters towards the objective gradually. 
There are a couple of possible objectives to learn in the literature; the state-value function, action-value function, or directly the policy.
In the case of the state value function, one can choose the objective function as the mean squared value loss between the estimated state-value function and the target state-value function (\ref{eqn:state-value}). While for policy approximation, expected return represents the objective function \ref{eqn:policy-objective}. 

Target state-value function can be the Monte-Carlo or TD target. Monte-Carlo’s target represents the expected return \((G_t)\), and TD target estimates the bootstrapped state-value function. The main difference between those targets is that Monte Carlo ensured to converge to a global optimum. In contrast, the bootstrapped TD target converges only to a local optimum near the global optimum. 

\begin{equation}
    \label{eqn:state-value}
    VE(w) = \sum\limits_{s\in S} \mu(s)[v_\pi(s) - \hat{v}(s,w)]^2
\end{equation}


Secondly, we need to optimize for the objective function. Optimaization is handled by the stochastic gradient descent method, it is the most used solution to improve the weight parameters based on a defined loss function. \ref{eqn:optimization} shows one gradient update step of the stochastic gradient descent method on the weight vector.  

\begin{equation}
    \label{eqn:optimization}
    w_{t+1} = w_t - \frac{1}{2}\alpha\nabla\Big[v_{\pi}(s) - \hat{v}(s,w)\Big]^2
\end{equation}

The optimal weight vector needs to find the right balance between strongly representing one state and generalizing to similar unseen states. As a result, our approximator needs to avoid either overfitting or underfitting.

% \subsection{Action Value Function Approximation}

One can also approximate action-value function in the same way as state-value function. The gradient update step of the action-value function is similar to the state-value function as in \ref{eqn:action_value}

\begin{equation}
    \label{eqn:action_value}
    w_{t+1} = w_t -\alpha\Big[U_t - \hat{q}(S_t,A_t, w_t)\Big]\nabla\hat{q}(S_t, A_t, w_t)
\end{equation}
\(U_t\) can be either TD or Monte Carlo target.

% \subsection{Policy Approximation}

So far, we have only considered the function approximation on the action and state functions.  Another promising and popular technique is to approximate the policy function directly. Policy function can lead us to optimal solution in a more direct way than action and state values. Since it directly aims to solve the problem at hand, to find the optimal policy \cite{SpinningUp2018}.
The objective function of policy gradient methods is simply the expectation of total return under policy \(\pi\).

\begin{equation}
    \label{eqn:policy-objective}
    J(\theta) = E\Big[\sum\limits_{t=0}^H R(s_t, u_t)\Big]
\end{equation}


\begin{equation}
    \theta \leftarrow \theta + \alpha\nabla_{\theta}J(\theta) 
\end{equation}