\subsection{Gazebo}

The Gazebo is the oldest simulator among Bullet and Mujoco. The development of Gazebo dates back to the 2002 University of Southern California. Later, Willow Garage took over action and extended Gazebo to ROS and PR2. Gazebo became the primary simulation engine of the ROS community. Eventually, in 2012 Gazebo became part of OSRF (Open Source Robotics Foundation), a spin-off from Willow Garage.

Technically, Gazebo is a simulation platform, which inherits different physics engines, such as Bullet, ODE, Simbody, and DART. According to the documentation of Gazebo 11.0, it supports ODE engine default, and other engines can be used, if developers compile Gazebo from the source. That means the performance of the overall Gazebo simulation highly depends on those individual physics engine’s performances.  Another dependency of Gazebo is ROS(Robotics Operating System). ROS infrastructure handles all communication. Thus, users need to rely on ROS to interact with Gazebo. We believe this assumption is too large. Even if relying on ROS can bring a lot of well-structured tool with it, learning ROS has a steep learning curve and has the potential to cause a large overhead for simple projects.
Nonetheless, developers invested highly on neat and clean documentation to reduce the overhead for new users. We consider the documentation and tutorials as a merit of the open-source project. Correspondingly, being open source contributes hugely to a large community willing to support, answering questions on forums, and submitting pull requests for possible bug fixes. 

According to Pitonakova et al., Gazebo has usability issues due to not having a 3D mesh editing option and difficulties of installing dependencies for 3rd party models. They also noted that Gazebo performs fairly well in large simulation environments, so it could be more convenient to conduct extensive swarm robotics experiments on Gazebo \cite{Pitonakova2018}.  

Based on our experiments in \ref{fig:GazeboPR2}, we found that Gazebo provides useful models to easily setup a table-top environment for robotics picking applications. Although we have not performed any grasping experiments on Gazebo, being able to edit the size, position, and orientation of models directly on simulation GUI is a useful feature, which lacks both on Pybullet and Mujoco.


\begin{figure}

    \begin{subfigure}{0.49\textwidth}
      \includegraphics[width=\linewidth]{figures/GazeboEnv1.png}
      \caption{} \label{fig:1a}
    \end{subfigure}%
    \hspace*{\fill}   % maximize separation between the subfigures
    \begin{subfigure}{0.49\textwidth}
      \includegraphics[width=\linewidth]{figures/GazeboEnv2.png}
      \caption{} \label{fig:1b}
    \end{subfigure}%

\caption{Table-top environment in Gazebo simulation with variety of objects and PR2 robot \label{fig:GazeboPR2}}
\end{figure}