% !TeX root = ../main.tex
% Add the above to each chapter to make compiling the PDF easier in some editors.

\chapter{Evaluation}\label{chapter:evaluation}

This section presents the experimental results of three different off-policy RL algorithms, BDQ, DQN, and SAC. Our initial goal by experimenting is to investigate the performance of the BDQ algorithm compared to its predecessor DQN and current state-of-art algorithm SAC on robotics grasping tasks. We want to understand how sensitive is the BDQ algorithm to hyperparameters and how well it explores the given problem. Moreover, most importantly, does the BDQ algorithm scale to high-dimensional action spaces? Through experiments, we aim to gather valuable insights into which kind of RL algorithm performs best on robotics grasping operation.

We tested our implementation of the BDQ algorithm based on a stable-baselines project \footnote{\url{https://github.com/BarisYazici/bdq_sb/tree/master/stable_baselines/bdq}}. All models are trained on the floor scene with a random-urdf object dataset. The RL agents' evaluation is based on the best performing model on the validation set in the floor scene. We examine the robustness of the models on the table scene.

Evaluation runs take 100 episodes long. Sequence, place, and shape of the objects are the same throughout different evaluation runs for different trials. In other words, all models are tested on identical conditions. An episode is considered successful when the gripper lifts an object to a predefined height.

We demonstrate the simplified environment's result and observe the differences between BDQ and DQN and compare it against SAC. Later, we test the same algorithms in the full environment description. We discern the scaling of the algorithms to a larger action space. Moreover, we looked at different perception pipelines' performance by varying the SAC algorithms input type from the encoder to RGBD or raw depth input. We check the importance of each module on the algorithm's success rate in the ablation studies section.  Ablation studies deliver the relative importance of curriculum strategy, normalization, actuator width, and reward to algorithm's overall performance. Finally, we provide specific failure cases for each algorithm


\section{Simplified Environment Results}

All three algorithms successfully converge to a decent grasping policy in the simplified environment. BDQ and DQN  were trained with varying action dimension padding 2, 4, 8, 16, and 33.
We observed how the neural network size affects the success rate. Two different network sizes were tried on BDQ; a large network with two hidden layers in shared module with 512 and 256 neurons each and 128 each for state and advantage function estimators, and a small network with two hidden layers in the shared representation with 64 neurons each and 32 neurons each for the state and advantage function estimators. Furthermore, we compared the BDQ performance with two different buffer sizes, fifty thousand and one million. We did not vary the  DQN's algorithm's hyperparameters. The default buffer size of 50k was used for all trials of DQN.
As for the SAC algorithm, we only change the input from the encoder to depth in the simplified environment.

\begin{table}[!htbp]
    \begin{tabular}{|l|l|l|l|l|}
    \hline
                       & \multicolumn{4}{c|}{\textbf{BDQ Simplified Scenario}}              \\ \hline
                       & \multicolumn{2}{c|}{Floor Scene} & \multicolumn{2}{c|}{Table Scene} \\ \hline
    Models             & Training Set      & Test Set     & Training Set      & Test Set     \\ \hline
    \textbf{BDQ\_33pads\_big}   & 34\%              & 29\%         & 13\%              & 12\%         \\ \hline
    \textbf{BDQ\_33pads\_small} & 90\%              & 82\%         & 86\%              & 79\%         \\ \hline
    \textbf{BDQ\_16pads\_small} & 89\%              & 93\%         & 90\%              & 91\%         \\ \hline
    \textbf{BDQ\_8pads\_small}  & 91\%              & 90\%         & 77\%              & 72\%         \\ \hline
    \textbf{BDQ\_4\_pads}       & 96\%              & 94\%         & 92\%              & 89\%         \\ \hline
    \end{tabular}
    \caption{BDQ algorithm's result in the simplified environment}
\end{table}


\begin{table}[!htbp]
    \begin{tabular}{|l|l|l|l|l|}
    \hline
                         & \multicolumn{4}{c|}{\textbf{DQN Simplified Scenario}}                                 \\ \hline
                         & \multicolumn{2}{c|}{\textbf{Floor Scene}} & \multicolumn{2}{c|}{\textbf{Table Scene}} \\ \hline
    \textbf{Models}      & \textbf{Training Set} & \textbf{Test Set} & \textbf{Training Set} & \textbf{Test Set} \\ \hline
    \textbf{DQN\_33pads} & 6\%                   & 6\%               & 6\%                   & 7\%               \\ \hline
    \textbf{DQN\_16pads} & 12\%                  & 6\%               & 9\%                   & 14\%              \\ \hline
    \textbf{DQN\_8pads}  & 75\%                  & 65\%              & 38\%                  & 40\%              \\ \hline
    \textbf{DQN\_4pads}  & 75\%                  & 73\%              & 1\%                   & 1\%               \\ \hline
    \textbf{DQN\_2pads}  & 76\%                  & 74\%              & 55\%                  & 63\%              \\ \hline
    \end{tabular}
    \caption{DQN algorithm's result in the simplified environment}
\end{table}



\begin{table}[!htbp]
    \begin{tabular}{|l|l|l|l|l|}
    \hline
                                    & \multicolumn{4}{c|}{\textbf{SAC Simplified  Scenario}}                         \\ \hline
                                    & \multicolumn{2}{c|}{\textbf{Floor Scene}} & \multicolumn{2}{c|}{\textbf{Table Scene}} \\ \hline
    \textbf{Models}                 & \textbf{Training Set}   & \textbf{Test Set}  & \textbf{Training Set} & \textbf{Test Set} \\ \hline
    \textbf{SAC\_encoder} & 98\%                 & 97\%               & 84\%                  & 87\%              \\ \hline
    \textbf{SAC\_depth}   & 91\%                 & 96\%               & 28\%                  & 24\%              \\ \hline
    \end{tabular}
    \caption{SAC algorithm's result in the simplified environment}
\end{table}

\begin{figure}[!htbp]
    \begin{subfigure}{0.49\textwidth}
        \includegraphics[width=\linewidth]{figures/BDQ_simplified_big_vs_small_network_no_var}
        \caption{Table Scene} \label{fig:table}
    \end{subfigure}%
    \hspace*{\fill}   % maximize separation between the subfigures
    \begin{subfigure}{0.49\textwidth}
        \includegraphics[width=\linewidth]{figures/DQN_vs_BDQ_in_simplified_env}
        \caption{Floor Scene} \label{fig:floor}
    \end{subfigure}%
    \hspace*{\fill}   % maximize separation between the subfigures


\caption{ Table and floor scenes \label{fig:scenes}}
\end{figure}

\begin{figure}[!htbp]
    \begin{subfigure}{0.49\textwidth}
        \includegraphics[width=\linewidth]{figures/DQN_simplified_env_increasing_action_dimension}
        \caption{Table Scene} \label{fig:table}
    \end{subfigure}%
    \hspace*{\fill}   % maximize separation between the subfigures
    \begin{subfigure}{0.49\textwidth}
        \includegraphics[width=\linewidth]{figures/BDQ_in_simplified_env_with_increasing_action_dimension}
        \caption{Floor Scene} \label{fig:floor}
    \end{subfigure}%
    \hspace*{\fill}   % maximize separation between the subfigures


\caption{ Table and floor scenes \label{fig:scenes}}
\end{figure}



\begin{figure}[!htbp]
    \begin{subfigure}{0.49\textwidth}
        \includegraphics[width=\linewidth]{figures/SAC_vs_BDQ_in_simplified_env}
        \caption{Table Scene} \label{fig:table}
    \end{subfigure}%
    \hspace*{\fill}   % maximize separation between the subfigures
    \begin{subfigure}{0.49\textwidth}
        \includegraphics[width=\linewidth]{figures/SAC_depth_vs_encoder_in_simplified_env}
        \caption{Floor Scene} \label{fig:floor}
    \end{subfigure}%
    \hspace*{\fill}   % maximize separation between the subfigures


\caption{ Table and floor scenes \label{fig:scenes}}
\end{figure}

% \begin{figure}[!htbp]
%     \centering
%         \includegraphics[width=0.4\textwidth]{figures/SAC_vs_BDQ_in_simplified_env}
%     \caption{Different manipulation skill adopted to robotic manipulators \cite{Kroemer2019}}
%     \label{fig:x manipulation_skills}
% \end{figure}

% \begin{figure}[!htbp]
%     \centering
%         \includegraphics[width=0.4\textwidth]{figures/SAC_depth_vs_encoder_in_simplified_env}
%     \caption{Different manipulation skill adopted to robotic manipulators \cite{Kroemer2019}}
%     \label{fig:x manipulation_skills}
% \end{figure}


\section{Full Environment Results}


In the full environment description, we generally evaluated the performance of variants of the SAC algorithm. We tested with different perception pipeline, different reward definition, and buffer size. Contrary to the simplified environment, we verified the algorithm's robustness on the wooden block object database.

We tested four variations of the SAC model in the full environment: encoder with 50k buffer size, encoder with 1m buffer size, depth with 1m buffer size, and RGBD with 1m buffer size. Among these models, the encoder 1m buffer achieved the best transfer result. It performed with 99\% and 82\% with random objects and wooden blocks in the table scene \ref{table:SACfull}. Nevertheless, training plots demonstrate that the encoder training process has more variance than the depth counterpart. Moreover, depth perception converges a higher success rate than encoder perception during training, as shown in the plot \ref{fig:percep}.

Like the simplified environment results, we also found that in the full environment, 1m buffer size performs better and accumulates less variance than the 50k buffer. As represented in the plot \ref{fig:encbuffer}, the 50k buffer version has more variance and converges to only a 90\% training success rate. In comparison, the 1m buffer size version subjected to less variance and converges around a 97\% success rate.  

Unexpectedly, depth perception also performs better than the RGBD. RGBD perception delivered the worst results among perception types. It fell to a 91\% test success rate even in the floor scene. In the same scene, depth perception achieved a 100\% test success rate. RGBD plot in figure \ref{fig:depthvsrgbd} also shows sudden changes which bring high variance. Nevertheless, it still managed to converge to a 99\% success rate.

Unfortunately, BDQ never delivered a working policy for the full environment. Its learning plot \ref{fig:bdq} also shows the severe decay to a 0\% success rate.

\begin{table}[!htbp]
    \begin{tabular}{l|l|l|l|l|}
    \cline{2-5}
                                  & \multicolumn{4}{c|}{\textbf{SAC Full Environment}}                                                                                                                                                                                                                                                                                                                                  \\ \cline{2-5}
                                  & \multicolumn{2}{c|}{\textbf{Floor Scene}}                                                                                                                                                & \multicolumn{2}{c|}{\textbf{Table Scene}}                                                                                                                                                \\ \hline
    \multicolumn{1}{|l|}{\textbf{Models}}               & \multicolumn{1}{c|}{\textbf{\begin{tabular}[c]{@{}c@{}}Random Obj. \\ (\%)\end{tabular}}} & \multicolumn{1}{c|}{\textbf{\begin{tabular}[c]{@{}c@{}}Wooden Obj.\\ (\%)\end{tabular}}} & \multicolumn{1}{c|}{\textbf{\begin{tabular}[c]{@{}c@{}}Random Obj. \\ (\%)\end{tabular}}} & \multicolumn{1}{c|}{\textbf{\begin{tabular}[c]{@{}c@{}}Wooden Obj.\\ (\%)\end{tabular}}} \\ \hline
    \multicolumn{1}{|l|}{\textbf{SAC\_encoder\_50k}}    & 65                                                                                          & 62                                                                                         & 63                                                                                          & 59                                                                                         \\ \hline
    \multicolumn{1}{|l|}{\textbf{SAC\_encoder\_1m}}     & 100                                                                                         & 95                                                                                         & 99                                                                                          & 82                                                                                         \\ \hline
    \multicolumn{1}{|l|}{\textbf{SAC\_depth}}           & 100                                                                                         & 95                                                                                         & 95                                                                                          & 23                                                                                         \\ \hline
    \multicolumn{1}{|l|}{\textbf{SAC\_rgbd}}            & 91                                                                                          & 95                                                                                         & 46                                                                                          & 17                                                                                         \\ \hline
    \multicolumn{1}{|l|}{\textbf{SAC\_depth\_sparse}}  & 99                                                                                          & 97                                                                                         & 100                                                                                         & 72                                                                                         \\ \hline
    \multicolumn{1}{|l|}{\textbf{SAC\_depth\_no\_curr}} & 100                                                                                         & 97                                                                                         & 97                                                                                          & 64                                                                                         \\ \hline
    \multicolumn{1}{|l|}{\textbf{SAC\_depth\_no\_act}}  & 95                                                                                          & 75                                                                                         & 84                                                                                          & 24                                                                                         \\ \hline
\end{tabular}
\caption{SAC Full environment results \label{table:SACfull}}
\end{table}


\begin{figure}[!htbp]
    \begin{subfigure}{0.49\textwidth}
        \includegraphics[width=\linewidth]{figures/SACfull/SAC_depth_vs_encoder_perception.png}
        \caption{SAC performance depth versus encoder perception} \label{fig:percep}
    \end{subfigure}%
    \hspace*{\fill}   % maximize separation between the subfigures
    \begin{subfigure}{0.49\textwidth}
        \includegraphics[width=\linewidth]{figures/SACfull/SAC_encoder_50k_vs_1m_buffer_size}
        \caption{SAC performance with encoder perception 50k vs 1m buffer} \label{fig:encbuffer}
    \end{subfigure}%
    \hspace*{\fill}   % maximize separation between the subfigures


\caption{ SAC performance comparison of perception pipelines and buffer size\label{fig:sacperf}}
\end{figure}

\begin{figure}[!htbp]
    \centering
        \includegraphics[width=0.7\textwidth]{figures/SACfull/SAC_vs_BDQ_full_environment}
    \caption{SAC compared to BDQ in full environment}
    \label{fig:bdq}
\end{figure}

\begin{figure}[!htbp]
    \centering
        \includegraphics[width=0.7\textwidth]{figures/SACfull/SAC_rgbd_vs_depth}
    \caption{SAC performance RGBD vs. Depth. Shaded region represents the variance.}
    \label{fig:depthvsrgbd}
\end{figure}

\section{Ablation Studies}

In ablation studies, we investigated the individual importance of the following modules: curriculum strategy, input and reward normalization, actuator-width information, and the shaped reward. All ablation study experiments were carried with the SAC algorithm with depth perception and one million experience replay buffer size. All training trials took place in the full environment floor scene. We shared the results of ablation studies in the same table as the SAC full environment \ref{table:SACfull}.

Among all ablation variants, only no normalization trial did not provide a working grasp policy. The rest either converged to a lower success rate or converged relatively slower than the baseline. 

Sparse reward and no curriculum learning converged 20 thousand timesteps and 400 thousand timesteps after the baselines. Interestingly, sparse reward SAC agent performed the best in the table scene, on random objects with a 100\% success rate. Both sparse and no curriculum learning trials performed better than the baseline SAC model on wooden blocks in the table scene. Sparse reward and no curriculum learning achieved 72\% and 64\%, while baseline reached 23\%.

No actuator width observation converged to a slightly worse success rate than the baseline.  It achieved a 97\% success rate, while the baseline converged to 99\%. An extra observation related to the environment proved to be useful.

\begin{figure}[htbp]
    \begin{subfigure}{0.49\textwidth}
        \includegraphics[width=\linewidth]{figures/ablation/SAC_performance_shaped_reward_vs_sparse_reward}
        \caption{SAC performance with shaped reward vs. sparse reward } \label{fig:table}
    \end{subfigure}%
    \hspace*{\fill}   % maximize separation between the subfigures
    \begin{subfigure}{0.49\textwidth}
        \includegraphics[width=\linewidth]{figures/ablation/SAC_performance_wo_actuator_width}
        \caption{SAC performance with encoder perception without actuator width observation} \label{fig:noact}
    \end{subfigure}%
    \hspace*{\fill}   % maximize separation between the subfigures


\caption{ Ablation of reward function and actuator width observation \label{fig:scenes}}
\end{figure}

\begin{figure}[htbp]
    \begin{subfigure}{0.49\textwidth}
        \includegraphics[width=\linewidth]{figures/ablation/SAC_performance_wo_curriculum_strategy}
        \caption{SAC performance withuot curriculum strategy} \label{fig:table}
    \end{subfigure}%
    \hspace*{\fill}   % maximize separation between the subfigures
    \begin{subfigure}{0.49\textwidth}
        \includegraphics[width=\linewidth]{figures/ablation/SAC_performance_wo_normalization}
        \caption{SAC performance in full environment without normalization compared to the performance of the baseline SAC} \label{fig:nonorm}
    \end{subfigure}%
    \hspace*{\fill}   % maximize separation between the subfigures


\caption{ Ablation of curriculum learning and normalization of input and reward \label{fig:scenes}}
\end{figure}







\section{DQN vs. BDQ}

\section{Encoder vs. Raw Sensor Input}

\section{Ablation Studies}

\subsection{Curriculum Strategy}

\subsection{Normalization}

\subsection{Shaped Reward}

\subsection{Actuator Width}

\section{Failure Modes}

